\protect\phantomsection\label{69fe235d}
\section{Modélisation du problème - Projet Green
Graph}\label{moduxe9lisation-du-probluxe8me---projet-green-graph}

\emph{Equipe CesiCDP - Chef de projet : Leila \textbar{} Opérateurs :
Tom, Edwin}

\subsection{Sommaire :}\label{sommaire-}

\begin{verbatim}
1. Introduction
2. Définition du problème 
    2.1. Description du problème de tournée de livraison
    2.2. Objectifs d’optimisation
3. Modélisation formelle
    3.1. Hypothèses générales
    3.2. Représentation mathématique du réseau
    3.3. Présentation des contraintes supplémentaires retenues
    3.4. Représentation formelle des contraintes choisies
4. Analyse de la complexité
    4.1. Classification du problème
    4.2. Impact des contraintes supplémentaires sur la complexité
5. Plan de Travail et Organisation du Projet
    5.1. Étapes prévues 
    5.2. Répartition des tâches dans l’équipe
    5.3. Outils utilisés 
6. Conclusion
7. Annexes
    7.1. Glossaire des termes techniques
    7.2. Références bibliographiques
\end{verbatim}

\protect\phantomsection\label{c757c557}
\subsubsection{1. Introduction}\label{1-introduction}

Depuis les années 90, la question environnementale s\textquotesingle est
imposé comme une priorité mondiale. Réduction de la consommation
d'énergie, diminution des émissions de gaz à effet de serre,
développement des mobilités durables\ldots{} de nombreux objectifs ont
été fixés, notamment via des accords internationaux (Protocole de Kyoto,
Accords de Paris) et des politiques nationales.

Dans ce cadre, L\textquotesingle ADEME (Agence de l'Environnement et de
la Maîtrise de l'Energie) avec la collaboration de CesiCDP, cherche à
optimiser la tournée de ses transports en commun afin de réduire ses
coûts financiers et son bilan carbone.

Comment concevoir une tournée de livraison sur un réseau routier reliant
plusieurs points, de manière à minimiser la consommation (temps,
distance, carburant), tout en respectant certaines contraintes
opérationnelles (capacités des véhicules, horaires de livraison, etc.) ?

Ce problème s'apparentant au problème du voyageur de commerce, est connu
pour sa complexité algorithmique. Cependant, il est applicable à de
nombreux domaines logistiques (livraison de colis, gestion des déchets,
interventions technique, etc.).\\
Les résultats obtenus pourront être adaptés à différentes tailles de
réseau et différents types de territoires (urbain, rural).

L\textquotesingle objectif sera de proposer dans un premier temps une
modélisation formelle du problème avec des représentations mathématiques
et une analyse de la complexité.\\
Par la suite une implémentation des algorithmes avec une démonstration
du fonctionnement de celle-ci sur différents cas de test.\\
On terminera par une étude expérimentale de nos solutions qui mettra en
évidence les perfomances et les limitations de nos algorithmes ainsi
qu\textquotesingle une perspective d\textquotesingle amélioration basé
sur notre analyse.

\paragraph{Contraintes}\label{contraintes}

\begin{longtable}[]{@{}
  >{\raggedright\arraybackslash}p{(\linewidth - 2\tabcolsep) * \real{0.5000}}
  >{\raggedright\arraybackslash}p{(\linewidth - 2\tabcolsep) * \real{0.5000}}@{}}
\toprule\noalign{}
\begin{minipage}[b]{\linewidth}\raggedright
Importance
\end{minipage} & \begin{minipage}[b]{\linewidth}\raggedright
Description
\end{minipage} \\
\midrule\noalign{}
\endhead
\bottomrule\noalign{}
\endlastfoot
Haute & Réduire au maximum la pollution émise par les véhicules \\
Haute & Optimiser les coûts de circulation (éviter les routes payantes,
les voies rapides) \\
Haute & Temps de calcul faible \\
Haute & Economie de ressources durant la génération de
l\textquotesingle itinéraire \\
Haute & Dépendance entre les visites \\
\end{longtable}

\protect\phantomsection\label{398a6f6d}
\subsubsection{2. Définition du
problème}\label{2-duxe9finition-du-probluxe8me}

\subparagraph{2.1 Description du problème de tournée de
livraison}\label{21-description-du-probluxe8me-de-tournuxe9e-de-livraison}

Une tournée de livraison telle que nous l\textquotesingle entendons fait
appel au probléme du voyageur de commerce, c\textquotesingle est à dire
que nous recherchons en un minimum de temps un itinéraire permettant de
repasser sur le moins de noeuds et d\textquotesingle arêtes possible.
Dans le cadre du projet, nous devons ajouter à ce problème des
contraites telles que le coup de passage sur certaines arêtes (ex:
autoroutes) ou encore la dépendance entre les visites (Une ville ne peut
être visitée qu\textquotesingle après en avoir visité une autre, une
livraison doit précéder une collecte).

\subparagraph{2.2. Objectifs
d'optimisation}\label{22-objectifs-doptimisation}

Afin d\textquotesingle optimiser la consommation de carburant et de
réduire les coûts, notre équipe a pour objectif de mettre en place un
algorithme qui prendra en compte les deux contraintes présentées
ci-dessus tout en choisissant un chemin le plus court possible. Nous
avons défini deux algorithmes de path finding,
l\textquotesingle algorithmes de A* et un algorithme génétique.

\protect\phantomsection\label{ec4a7404}
\subsubsection{3. Modélisation
formelle}\label{3-moduxe9lisation-formelle}

\subparagraph{3.1. Hypothèses
générales}\label{31-hypothuxe8ses-guxe9nuxe9rales}

\subparagraph{3.2. Représentation mathématique du
réseau}\label{32-repruxe9sentation-mathuxe9matique-du-ruxe9seau}

Le réseau routier francais peut être représenté par un graphe pondéré
connexe et complet \(G = (V,E) V\) représentant chacune des villes
francaises et \(E\) les routes entre chacune de ces villes. La
pondération de chaque arêtes entre deux sommets \(A\) \& \(B\) et
réalisée grâce à la formule suivante : (A définir)

\subparagraph{3.3. Présentation des contraintes supplémentaires
retenues}\label{33-pruxe9sentation-des-contraintes-suppluxe9mentaires-retenues}

\subparagraph{3.4. Représentation formelle des contraintes
choisies}\label{34-repruxe9sentation-formelle-des-contraintes-choisies}

\protect\phantomsection\label{6aa48521}
\subsubsection{4. Analyse de la
complexité}\label{4-analyse-de-la-complexituxe9}

\subparagraph{4.1. Classification du
problème}\label{41-classification-du-probluxe8me}

\subparagraph{\texorpdfstring{\textbf{Démonstration de la NP-Complétude
du Problème du Voyageur de Commerce
(TSP)}}{Démonstration de la NP-Complétude du Problème du Voyageur de Commerce (TSP)}}\label{duxe9monstration-de-la-np-compluxe9tude-du-probluxe8me-du-voyageur-de-commerce-tsp}

\subparagraph{\texorpdfstring{\textbf{Construction de
l\textquotesingle instance de
TSP}}{Construction de l\textquotesingle instance de TSP}}\label{construction-de-linstance-de-tsp}

Soit \((I_{CH})\) une instance du Cycle Hamiltonien sur un graphe
\((G = (V, E))\). Nous construisons une instance \((I_{TSP})\) de TSP
comme suit :

\begin{enumerate}
\tightlist
\item
  \textbf{Graphe \((G' = (V, E'))\)} :

  \begin{itemize}
  \tightlist
  \item
    \((E')\) contient toutes les arêtes de \((G)\) avec un coût de
    \textbf{1}.
  \item
    Les arêtes absentes de \((G)\) (i.e., celles de son complément
    \((\overline{G})\)) sont ajoutées à \((E')\) avec un coût de
    \textbf{2}.
  \end{itemize}
\item
  \textbf{Valeur \((k = |V|)\)} : La longueur maximale autorisée pour le
  cycle TSP est fixée à \((|V|)\).
\end{enumerate}

Cette construction s\textquotesingle effectue en temps polynomial (en
\((O(|V|^2))\)).

\subparagraph{\texorpdfstring{\textbf{Réduction et Preuve
d\textquotesingle Équivalence}}{Réduction et Preuve d\textquotesingle Équivalence}}\label{ruxe9duction-et-preuve-duxe9quivalence}

\begin{enumerate}
\tightlist
\item
  \textbf{Si \((G)\) a un Cycle Hamiltonien} :

  \begin{itemize}
  \tightlist
  \item
    Ce cycle utilise exactement \((|V|)\) arêtes de \((G)\), toutes de
    coût \textbf{1}.
  \item
    Dans \((I_{TSP})\), ce cycle correspond à un circuit de coût total
    \((|V|)\), répondant "oui" à \((I_{TSP})\).
  \end{itemize}
\item
  \textbf{Si \((I_{TSP})\) répond "oui"} :

  \begin{itemize}
  \tightlist
  \item
    Le circuit TSP a un coût \((\leq |V|)\). Comme les arêtes de
    \((\overline{G})\) coûtent \textbf{2}, le circuit ne peut en
    utiliser aucune (sinon le coût total dépasserait \((|V|)\)).
  \item
    Le circuit utilise donc uniquement des arêtes de \((G)\), formant un
    Cycle Hamiltonien dans \((G)\).
  \end{itemize}
\item
  \textbf{Si \((G)\) n\textquotesingle a pas de Cycle Hamiltonien} :

  \begin{itemize}
  \tightlist
  \item
    Tout circuit TSP dans \((G')\) doit utiliser au moins une arête de
    \((\overline{G})\), ce qui entraîne un coût \((\geq |V| + 1)\).
  \item
    \((I_{TSP})\) répond donc "non".
  \end{itemize}
\end{enumerate}

\subparagraph{\texorpdfstring{\textbf{Conclusion}}{Conclusion}}\label{conclusion}

\begin{itemize}
\tightlist
\item
  \textbf{TSP est NP-Difficile} : Toute instance de HC (NP-Complet) se
  réduit polynomialement à TSP.
\end{itemize}

Ainsi, TSP appartient à la classe \textbf{NP-Complet}. Cette réduction
exploite la structure des coûts dans \(( G')\) en liant le probleme du
Cycle Hamiltonien à une solution optimale de TSP.

\subparagraph{4.2. Impact des contraintes supplémentaires sur la
complexité}\label{42-impact-des-contraintes-suppluxe9mentaires-sur-la-complexituxe9}

\protect\phantomsection\label{d856fa45}
\subsubsection{5. Plan de Travail et Organisation du
Projet}\label{5-plan-de-travail-et-organisation-du-projet}

\subparagraph{5.1. Étapes prévues}\label{51-uxe9tapes-pruxe9vues}

\begin{longtable}[]{@{}
  >{\raggedright\arraybackslash}p{(\linewidth - 2\tabcolsep) * \real{0.5000}}
  >{\raggedright\arraybackslash}p{(\linewidth - 2\tabcolsep) * \real{0.5000}}@{}}
\toprule\noalign{}
\begin{minipage}[b]{\linewidth}\raggedright
Date de livraison
\end{minipage} & \begin{minipage}[b]{\linewidth}\raggedright
Description
\end{minipage} \\
\midrule\noalign{}
\endhead
\bottomrule\noalign{}
\endlastfoot
\textbf{09/04/2025 - 15/04/2025} & \textbf{Livrable 1} \\
09/04/2025 - 09/04/2025 & Reformulation de la problématique \\
09/04/2025 - 12/04/2025 & Calcul de complexité \\
10/04/2025 - 14/04/2025 & Représentation formelle des données \\
10/04/2025 - 12/04/2025 & Représentation formelle des problèmes \\
13/04/2025 - 15/04/2025 & Représentation formelle des objectifs à
optimiser \\
\textbf{15/04/2025 - 28/04/2025} & \textbf{Livrable 2} \\
15/04/2025 - 28/04/2025 & Mise à jour de la modélisation du livrable
1 \\
15/04/2025 - 16/04/2025 & Décrit les méthodes de résolution choisies :
détails sur les algorithmes utilisés \\
15/04/2025 - 17/04/2025 & L'implémentation de ces algorithmes \\
16/04/2025 - 18/04/2025 & Etude expérimentale : Plan
d\textquotesingle expérience \\
18/04/2025 - 19/04/2025 & Etude expérimentale : benchmarks \\
20/04/2025 - 24/04/2025 & Etude expérimentale : limitations /
améliorations possibles \\
\end{longtable}

\subparagraph{5.2. Outils utilisés}\label{52-outils-utilisuxe9s}

\begin{itemize}
\tightlist
\item
  VSCode
\item
  Jupyter notebook
\item
  Git/Github
\item
  Office
\end{itemize}

Langages et librairies :

\begin{itemize}
\tightlist
\item
  Python 3.12
\item
  Package de lib pour OpenStreetMap
\item
  matplotlib
\item
  numpy
\item
  ipywidgets
\end{itemize}

\protect\phantomsection\label{0dc81c2e}
\subsubsection{6. Conclusion}\label{6-conclusion}

\protect\phantomsection\label{f2e9cf69}
\subsubsection{7. Annexes}\label{7-annexes}

\subparagraph{7.1. Glossaire des termes
techniques}\label{71-glossaire-des-termes-techniques}

\begin{itemize}
\item
  NP, NP-Difficile, NP-Complet : En informatique théorique, et plus
  précisément en théorie de la complexité, une classe de complexité est
  un ensemble de problèmes algorithmiques dont la résolution nécessite
  la même quantité d\textquotesingle une certaine ressource.

  \subparagraph{7.2. Références
  bibliographiques}\label{72-ruxe9fuxe9rences-bibliographiques}
\end{itemize}

\protect\phantomsection\label{7047559e}
\subsubsection{5. Données de Test}\label{5-donnuxe9es-de-test}

Pour répondre à ce probleme nous utiliserons les villes de france obtenu
grace à World City DB ,

\protect\phantomsection\label{c0f3a82f}
\begin{Shaded}
\begin{Highlighting}[]
\ImportTok{import}\NormalTok{ requests}
\ImportTok{import}\NormalTok{ json}
\ImportTok{from}\NormalTok{ geopy.geocoders }\ImportTok{import}\NormalTok{ Nominatim}

\KeywordTok{def}\NormalTok{ geocode\_city(city\_name):}
    \CommentTok{"""Converts a city name to coordinates (latitude, longitude)"""}
\NormalTok{    geolocator }\OperatorTok{=}\NormalTok{ Nominatim(user\_agent}\OperatorTok{=}\StringTok{"routing\_app"}\NormalTok{)}
\NormalTok{    location }\OperatorTok{=}\NormalTok{ geolocator.geocode(city\_name)}
    
    \ControlFlowTok{if}\NormalTok{ location:}
        \ControlFlowTok{return}\NormalTok{ (location.latitude, location.longitude)}
    \ControlFlowTok{else}\NormalTok{:}
        \ControlFlowTok{raise} \PreprocessorTok{ValueError}\NormalTok{(}\SpecialStringTok{f"Unable to find coordinates for }\SpecialCharTok{\{}\NormalTok{city\_name}\SpecialCharTok{\}}\SpecialStringTok{"}\NormalTok{)}

\KeywordTok{def}\NormalTok{ calculate\_travel\_time(departure\_city, arrival\_city, mode}\OperatorTok{=}\StringTok{"driving"}\NormalTok{):}
    \CommentTok{"""}
\CommentTok{    Calculates travel time between two cities using the OSRM API}
\CommentTok{    }
\CommentTok{    Args:}
\CommentTok{        departure\_city (str): Name of the departure city}
\CommentTok{        arrival\_city (str): Name of the arrival city}
\CommentTok{        mode (str): Transportation mode (driving, cycling, walking)}
\CommentTok{        }
\CommentTok{    Returns:}
\CommentTok{        tuple: (time in seconds, distance in meters, formatted time)}
\CommentTok{    """}
    \CommentTok{\# Get city coordinates}
    \ControlFlowTok{try}\NormalTok{:}
\NormalTok{        lat1, lon1 }\OperatorTok{=}\NormalTok{ geocode\_city(departure\_city)}
\NormalTok{        lat2, lon2 }\OperatorTok{=}\NormalTok{ geocode\_city(arrival\_city)}
    \ControlFlowTok{except} \PreprocessorTok{ValueError} \ImportTok{as}\NormalTok{ e:}
        \ControlFlowTok{return}\NormalTok{ (}\VariableTok{None}\NormalTok{, }\VariableTok{None}\NormalTok{, }\BuiltInTok{str}\NormalTok{(e))}
    
    \CommentTok{\# Building the URL for the OSRM API}
\NormalTok{    url }\OperatorTok{=} \SpecialStringTok{f"http://router.project{-}osrm.org/route/v1/}\SpecialCharTok{\{}\NormalTok{mode}\SpecialCharTok{\}}\SpecialStringTok{/}\SpecialCharTok{\{}\NormalTok{lon1}\SpecialCharTok{\}}\SpecialStringTok{,}\SpecialCharTok{\{}\NormalTok{lat1}\SpecialCharTok{\}}\SpecialStringTok{;}\SpecialCharTok{\{}\NormalTok{lon2}\SpecialCharTok{\}}\SpecialStringTok{,}\SpecialCharTok{\{}\NormalTok{lat2}\SpecialCharTok{\}}\SpecialStringTok{"}
\NormalTok{    params }\OperatorTok{=}\NormalTok{ \{}
        \StringTok{"overview"}\NormalTok{: }\StringTok{"false"}\NormalTok{,}
        \StringTok{"alternatives"}\NormalTok{: }\StringTok{"false"}\NormalTok{,}
\NormalTok{    \}}
    
    \CommentTok{\# Call to the OSRM API}
\NormalTok{    response }\OperatorTok{=}\NormalTok{ requests.get(url, params}\OperatorTok{=}\NormalTok{params)}
    
    \ControlFlowTok{if}\NormalTok{ response.status\_code }\OperatorTok{!=} \DecValTok{200}\NormalTok{:}
        \ControlFlowTok{return}\NormalTok{ (}\VariableTok{None}\NormalTok{, }\VariableTok{None}\NormalTok{, }\SpecialStringTok{f"Error during API call: }\SpecialCharTok{\{}\NormalTok{response}\SpecialCharTok{.}\NormalTok{status\_code}\SpecialCharTok{\}}\SpecialStringTok{"}\NormalTok{)}
    
\NormalTok{    data }\OperatorTok{=}\NormalTok{ response.json()}
    
    \ControlFlowTok{if}\NormalTok{ data[}\StringTok{"code"}\NormalTok{] }\OperatorTok{!=} \StringTok{"Ok"}\NormalTok{:}
        \ControlFlowTok{return}\NormalTok{ (}\VariableTok{None}\NormalTok{, }\VariableTok{None}\NormalTok{, }\SpecialStringTok{f"OSRM API error: }\SpecialCharTok{\{}\NormalTok{data[}\StringTok{\textquotesingle{}code\textquotesingle{}}\NormalTok{]}\SpecialCharTok{\}}\SpecialStringTok{"}\NormalTok{)}
    
    \CommentTok{\# Extracting time and distance information}
\NormalTok{    route }\OperatorTok{=}\NormalTok{ data[}\StringTok{"routes"}\NormalTok{][}\DecValTok{0}\NormalTok{]}
\NormalTok{    duration\_seconds }\OperatorTok{=}\NormalTok{ route[}\StringTok{"duration"}\NormalTok{]}
\NormalTok{    distance\_meters }\OperatorTok{=}\NormalTok{ route[}\StringTok{"distance"}\NormalTok{]}
    
    \CommentTok{\# Formatting time for display}
\NormalTok{    hours, remainder }\OperatorTok{=} \BuiltInTok{divmod}\NormalTok{(duration\_seconds, }\DecValTok{3600}\NormalTok{)}
\NormalTok{    minutes, seconds }\OperatorTok{=} \BuiltInTok{divmod}\NormalTok{(remainder, }\DecValTok{60}\NormalTok{)}
    
\NormalTok{    formatted\_time }\OperatorTok{=} \StringTok{""}
    \ControlFlowTok{if}\NormalTok{ hours }\OperatorTok{\textgreater{}} \DecValTok{0}\NormalTok{:}
\NormalTok{        formatted\_time }\OperatorTok{+=} \SpecialStringTok{f"}\SpecialCharTok{\{}\BuiltInTok{int}\NormalTok{(hours)}\SpecialCharTok{\}}\SpecialStringTok{ hour}\SpecialCharTok{\{}\StringTok{\textquotesingle{}s\textquotesingle{}} \ControlFlowTok{if}\NormalTok{ hours }\OperatorTok{\textgreater{}} \DecValTok{1} \ControlFlowTok{else} \StringTok{\textquotesingle{}\textquotesingle{}}\SpecialCharTok{\}}\SpecialStringTok{ "}
    \ControlFlowTok{if}\NormalTok{ minutes }\OperatorTok{\textgreater{}} \DecValTok{0}\NormalTok{:}
\NormalTok{        formatted\_time }\OperatorTok{+=} \SpecialStringTok{f"}\SpecialCharTok{\{}\BuiltInTok{int}\NormalTok{(minutes)}\SpecialCharTok{\}}\SpecialStringTok{ minute}\SpecialCharTok{\{}\StringTok{\textquotesingle{}s\textquotesingle{}} \ControlFlowTok{if}\NormalTok{ minutes }\OperatorTok{\textgreater{}} \DecValTok{1} \ControlFlowTok{else} \StringTok{\textquotesingle{}\textquotesingle{}}\SpecialCharTok{\}}\SpecialStringTok{"}
    
    \ControlFlowTok{return}\NormalTok{ (duration\_seconds, distance\_meters, formatted\_time.strip())}

\KeywordTok{def}\NormalTok{ display\_route(departure\_city, arrival\_city, mode}\OperatorTok{=}\StringTok{"driving"}\NormalTok{):}
    \CommentTok{"""Displays route information between two cities"""}
\NormalTok{    modes }\OperatorTok{=}\NormalTok{ \{}
        \StringTok{"driving"}\NormalTok{: }\StringTok{"by car"}\NormalTok{,}
        \StringTok{"cycling"}\NormalTok{: }\StringTok{"by bicycle"}\NormalTok{,}
        \StringTok{"walking"}\NormalTok{: }\StringTok{"on foot"}
\NormalTok{    \}}
    
\NormalTok{    duration, distance, message }\OperatorTok{=}\NormalTok{ calculate\_travel\_time(departure\_city, arrival\_city, mode)}
    
    \ControlFlowTok{if}\NormalTok{ duration }\KeywordTok{is} \VariableTok{None}\NormalTok{:}
        \BuiltInTok{print}\NormalTok{(message)}
        \ControlFlowTok{return}
    
    \BuiltInTok{print}\NormalTok{(}\SpecialStringTok{f"Route from }\SpecialCharTok{\{}\NormalTok{departure\_city}\SpecialCharTok{\}}\SpecialStringTok{ to }\SpecialCharTok{\{}\NormalTok{arrival\_city}\SpecialCharTok{\}}\SpecialStringTok{ }\SpecialCharTok{\{}\NormalTok{modes}\SpecialCharTok{.}\NormalTok{get(mode, }\StringTok{\textquotesingle{}\textquotesingle{}}\NormalTok{)}\SpecialCharTok{\}}\SpecialStringTok{:"}\NormalTok{)}
    \BuiltInTok{print}\NormalTok{(}\SpecialStringTok{f"Travel time: }\SpecialCharTok{\{}\NormalTok{message}\SpecialCharTok{\}}\SpecialStringTok{"}\NormalTok{)}
    \BuiltInTok{print}\NormalTok{(}\SpecialStringTok{f"Distance: }\SpecialCharTok{\{}\NormalTok{distance}\OperatorTok{/}\DecValTok{1000}\SpecialCharTok{:.1f\}}\SpecialStringTok{ km"}\NormalTok{)}

\end{Highlighting}
\end{Shaded}

\begin{verbatim}
---------------------------------------------------------------------------
ModuleNotFoundError                       Traceback (most recent call last)
Cell In[11], line 1
----> 1 import requests
      2 import json
      3 from geopy.geocoders import Nominatim

ModuleNotFoundError: No module named 'requests'
\end{verbatim}

\protect\phantomsection\label{3998430a}

\protect\phantomsection\label{69900ff1}
\begin{Shaded}
\begin{Highlighting}[]
\BuiltInTok{print}\NormalTok{(display\_route(}\StringTok{"Reims"}\NormalTok{, }\StringTok{"Paris"}\NormalTok{, }\StringTok{"driving"}\NormalTok{))}
\end{Highlighting}
\end{Shaded}

\begin{verbatim}
Route from Reims to Paris by car:
Travel time: 1 hour 40 minutes
Distance: 144.0 km
None
\end{verbatim}

\protect\phantomsection\label{2e4ae18a}
\begin{Shaded}
\begin{Highlighting}[]

\end{Highlighting}
\end{Shaded}
